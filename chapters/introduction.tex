\chapter{Introduction}
\label{introduction}

Graphs and networks are among the most basic mathematical objects known, yet
many questions about graphs are open. In particular, many notions that are well
understood in simple domains become much more nuanced and complicated when
generalized to networks. The added complexity comes from the interplay between
the structure of the graph and the mathematical complexities of the problem. As
an example from game theory, two-player games are well understood, but
$n$-player games played on a network are often much more complicated. Indeed,
we provide concrete evidence of this in Chapter~\ref{ch:anticoordination}. 

Networks are also becoming an increasingly important part of how computer
science is applied more generally. As a short list, the algorithmic study of
networks is applied to the internet, online social networks, routing networks,
computing networks, power grids, biological networks, and many others.
Moreover, real world networks scale to millions or billions of vertices. This
gives rise to three needs in the study of networks:

\begin{enumerate}
\item Understanding the asymptotic complexity of algorithms on networks, and
moreover understanding how the specific properties of networks in a problem
domain differ from general networks. We explore this in
Chapter~\ref{ch:resilience}.
\item Understanding the capability of distributed computing models to solve
problems on massive networks. We explore this theme in
Chapter~\ref{ch:mapreduce}.  
\item When problems are known to be intractable, understanding the potential
for one to find an approximate solution. We study this for a specific problem
in Chapter~\ref{ch:nanip}.
\end{enumerate}

Many of these needs interleave with the goal of understanding what makes a
graph problem computationally intractable, and of finely delineating the
boundary between tractable and intractable regimes. That theme unites this
dissertation.  We now give a short summary of the contributions presented in
this dissertation, followed by background information for each chapter.

In the Chapter~\ref{ch:anticoordination} we discuss \emph{anti-coordination}
games played on graphs. These are games in which players (nodes in a graph) are
incentivized to choose strategies that differ from their neighbors. We
characterize the price of anarchy for these games, which measures the tradeoff
between players acting independently and greedily versus a central planning
authority. We introduce a directed graph generalization which allows one to
model both anti-coordination and coordination incentives. We further prove that
the complexity of computing strategies with certain properties (akin to being a
certain kind of Nash equilibrium) is NP-hard.

In Chapter~\ref{ch:resilience} we introduce a new model for measuring the
complexity of a combinatorial decision problem called \emph{resilience}.
Loosely speaking, an instance of a problem is resilient if it is a
``yes-instance'' (i.e. it has the combinatorial property),  and remains so
under small adversarial manipulations. Then the goal is to distinguish between
``no-instances'' and resilient instances. For graph coloring, this corresponds
to a graph which is, say, 3-colorable and remains so even after an adversary
adds an arbitrary edge to the graph. In general, we ask how resilient a problem
must be in order to make finding solutions tractable.  Surprisingly, for the
example of graph 3-coloring with the ability to add a single arbitrary edge, it
remains NP-hard to find a 3-coloring. We further study the gradient between
hardness and tractability for resilient coloring. We also completely
characterize the complexity of resilient boolean satisfiability: it is either
vacuous or NP-hard.

In Chapter~\ref{ch:mapreduce} we turn our attention to \emph{MapReduce}, a
popular model of distributed computation which has novel constraints on
communication and space. We first refine an existing theoretical model of
MapReduce. Then we prove a general result on the ability of a two-round
MapReduce protocol to capture all of sublogarithmic space Turing machines.
Finally, we prove a connection between MapReduce, the exponential time
hypothesis (ETH), and long-standing open conjectures about complexity
hierarchies within simultaneous time/space-bounded complexity classes (TISP).
In particular, the exponential time hypothesis implies a hierarchy within
linear-space TISP, which in turn implies a hierarchy within MapReduce for each
of the parameters of interest.

In Chapter~\ref{ch:nanip}, we study the Neighbor Aided Network Installation
Problem (NANIP), which models disaster recovery in a network. This problem asks
one to determine the optimal ordering of nodes in a graph, not necessarily as a
path, which minimizes the cost of traversing the nodes in that order. The cost
of traversing a node is a function of the number of neighbors that have already
been visited. The chapter presents three contributions. First, we prove that
NANIP is NP-hard even under convex cost functions, has no FPTAS, and more
generally cannot be approximated to within a factor of $(1-n^{-c})$ for all
$c>0$. Second, we disprove a conjecture of Gutfraind et al.~\cite{Gutfraind14}
on the optimality of the greedy algorithm for ``connected'' solutions of NANIP,
instead showing that no connected algorithm can approximate NANIP to within a
logarithmic multiplicative factor. The greedy algorithm specifically has a
linear approximation lower bound. Third, we develop a new integer programming
formulation of NANIP by adapting the formulation of Miller, Tucker, and
Zemlin~\cite{miller1960integer}, and measure the improvement over the state of
the art.~\cite{Gutfraind14}
 
\section{Basic Definitions}

An \emph{undirected graph} $G = (V,E)$ consists of a set of vertices $V$ and a
set of unordered pairs called edges $E \subset V \times V$. A \emph{directed}
graph gives orientation to the edges, i.e. the edge set $E$ are instead ordered
pairs in $V \times V$. The \emph{degree} of a vertex in an undirected graph is
$\textup{deg}_G(v) = |\{e \in E : v \in e \}|$, with \emph{in-degree} and
\emph{out-degree} defined analogously for directed graphs. Sometimes when there
are multiple graphs we will specify by denoting $V = V(G), E = E(G)$. A
\emph{path} is an alternating list $(v_1, e_1, v_2, e_2, \dots, e_{k-1}, v_k)$
where $\forall i, e_i = (v_i,v_{i+1})$. An undirected graph is \emph{connected}
if every pair of nodes $v,w \in V$ has a path connecting $v$ to $w$.

A graph $G = (V,E)$ is called $k$-colorable if there is an assignment $f:V \to \{
1, 2, \dots, k \}$ such that for every edge $e = (u,v)$, $f(u) \neq f(v)$. I.e.
if $\{1,\dots, k\}$ are thought of as colors, then no edge is monochromatic.
The \emph{chromatic number} of a graph, $\chi(G)$, is the smallest integer $k$
for which $G$ is $k$-colorable. The \emph{complete} graph $K_n$ is the
$n$-vertex graph which has an edge between every pair of vertices. The CLIQUE
problem is the problem of determining, given an undirected graph $G$ and an
integer $m$ as input, whether $G$ contains a subgraph isomorphic to $K_m$.

The problem of \emph{boolean satisfiability} is the decision problem asking for
a given propositional formula $\varphi$, whether some assignment of its
variables makes $\varphi$ true. If there is such an assignment, the formula is
said to be \emph{satisfiable}. A formula $\varphi$ is said to be in
\emph{conjunctive normal form} with clauses of size $k$, or $k$-CNF form, if it
can be written as $\varphi = C_1 \wedge \dots \wedge C_m$, where each $C_i$ is
a disjunction of three literals. Boolean satisfiability for $k$-CNF formulas is
called $k$-satisfiability, or $k$-SAT. 

As 3-SAT is a classical NP-hard problem~\cite{GareyJ79}, we use it to prove the
hardness of many problems (provided P $\neq$ NP) via so-called \emph{gadget
reductions}. A \emph{polynomial-time} reduction from problem (language) $A$ to
problem (language) $B$ is a polynomial-time computable function $f : \{0,1\}^*
\to \{0,1\}^*$ such that $x \in A$ if and only if $f(x) \in B$. That is, $x$
has a ``yes'' answer if and only if $f(x)$ has a ``yes'' answer. A gadget
reduction is a type of reduction that is \emph{local} in the sense that each
bit of the output $f(x)$ can only depend on a bounded number of bits of the
input $x$ (rigorously, $f$ is computable by constant-depth NC circuits). Less
formally, a gadget reduction from 3-SAT to a graph problem requires one to, for
each formula $\varphi$, construct a graph $G$ via a collection of subgraphs
and pick an interpretation of truth/falsity such that 

\begin{enumerate}
\item Some of the subgraphs correspond to literals of $\varphi$ (literal
gadget).
\item Some of the subgraphs ensure that negated literals from $\varphi$ have
negated interpretations in $G$ (negation gadget).
\item Some of the subgraphs correspond to clauses (clause gadget).
\item The interpretation for a clause gadget to be true is satisfied if and
only if the interpretation for the truth of one of the member literal gadgets
is satisfied.
\end{enumerate}

We do this in both Chapters~\ref{ch:anticoordination} and~\ref{ch:resilience},
and design a new type of literal gadget for reducing SAT problems to coloring
problems, where a literal is represented by a pair of vertices and is
considered ``true'' if the two vertices have the same color.

\section{Algorithmic Game Theory}

Game theory generally studies strategies for competitive scenarios (games), and
the properties of various kinds of equilibria. In particular, a game theorist
typically studies game-theoretic models for real life scenarios, as well as the
existence and uniqueness of a strategy that is `optimal' or `stable' in some
sense.

A classical theorem from game theory called \emph{fictitious
play}~\cite{Robinson51} gives an algorithmic method to find an equilibrium
strategy in a variety of settings~\cite{Berger05,MondererS96,Nachbar90}.
However, this algorithm is only guaranteed to converge, and may take
exponentially long to do so~\cite{DaskalakisGP09}. Algorithmic game theory
distinguishes itself in part from classical game theory in that it places a
major emphasis on the computational complexity of finding optimal equilibrium
strategies. Significant progress in this area has been in the definition of the
PPAD complexity class~\cite{Papadimitriou94}, which both encapsulates the
complexity of computing Nash equilibria in two-player games and connects this
to complexity questions in classical mathematical settings, such as computing
Brouwer fixed points~\cite{ChenD09}. The results in
Chapter~\ref{ch:anticoordination} are related to the flip-side of this
question, when equilibria may not exist. For most of the cases we consider, the
complexity of deciding whether there is an equilibrium is NP-hard.

For our anti-coordination game, it turns out that finding \emph{strict}
equilibria (where a player will necessarily perform worse by deviating) is
NP-hard, as well as finding any equilibrium in a directed graph. Our
NP-hardness results are gadget reductions from boolean 3-satisfiability
(3-SAT), the problem of determining whether a propositional formula in
conjunctive normal form (with clauses of size 3) has a satisfying assignment. A
useful tool we use in these results is to represent each literal by a pair of
vertices, calling the literal ``true'' if the two agents represented by that
literal choose the same strategy. This technique is also used in
Chapter~\ref{ch:resilience} to reduce from ``resilient'' satisfiability to
resilient 3-coloring. 

Another important angle is the measure of efficiency of equilibria in games
with many players. The central concern is that an equilibrium that players
naturally reach may result in a much lower social welfare than a
non-equilibrium strategy, such as in a prisoner's dilemma. Here \emph{social
welfare} is defined as the sum of the payoffs for all players, which is only
interesting for games which are not zero-sum. The quantity that measures
efficiency is called the \emph{price of anarchy}~\cite{KP99}, and it measures
the ratio between the social welfare of the best equilibrium and the maximal
social welfare under any strategy. We compute the price of anarchy for our
anti-coordination game, which approaches 1 as the number of players grows. Our
computation and setting was later generalized in\cite{FeldmanF15}.

Finally, the method of a potential function is an important technique that we
use to prove the existence of an equilibrium (and implicitly an algorithm for
computing one) in the simplest setting studied in
Chapter~\ref{ch:anticoordination}. The idea here is to construct valuation
$\varphi$ on the strategy profile of the players that satisfies two properties.
First, when players modify their strategies over time (imagining that they are
playing a repeated game) the value of $\varphi$ provably increases, in our case
monotonically. Second, a local maximum of $\varphi$ corresponds to the desired
type of equilibrium. Games which lend themselves to such analyses are called
\emph{potential games}, and these have been extensively studied in the game
theory literature. See, e.g.,~\cite{MondererS96} for a generic classification
theorem.

\section{Graph Coloring and Resilience}

One of the first important things one learns when studying graph theory is that
graph coloring is hard. Recall that coloring a graph with $k$ colors assigning
each vertex a color (a number in $\{ 1, 2, \dots, k \}$) so that no edge is
monochromatic. Deciding whether a graph can be colored with $k$ colors for any
$k \geq 3$ has no known polynomial time algorithm and is a classical NP-hard
problem~\cite{GareyJ79}.  

One might naturally think graph coloring has a gradient of difficulty. Perhaps,
as graphs get more ``complex'' it becomes algorithmically harder to figure out
how colorable they are. There are many well-known notions of simplicity for
graphs, but they rarely fall on a gradient. For example, here are some ways to
make graph coloring easy:

\begin{itemize}
\item Restrict to planar graphs. Then deciding 4-colorability is easy because
the answer is always yes.~\cite{AppelH77I,AppelH77II}

\item Restrict to triangle-free planar graphs. Then finding a 3-coloring is
in P. (There are many algorithms, see e.g.~\cite{DvovrakKT11}.)

\item Restrict to perfect graphs (which again requires knowledge about how
colorable it is).\cite{GrotschelLS12,HMM10}

\item Restrict to graphs of tree-width or clique-width bounded by a
constant.~\cite{Cai03}

\item Restrict to graphs that are characterized by omitting a certain kind of
induced subgraph (such as having no induced paths of length 4 or
5).~\cite{KKTW01}
\end{itemize}

It should be emphasized that these results are very difficult to compare.  The
properties are inherently binary (either perfect or imperfect, planar or not
planar). Coloring general graphs is much bleaker, where the focus has turned to
approximations. A typical goal for an approximation algorithm would be to find
an algorithm that can color a graph $G$ which has true chromatic number
$\chi(G)$ using at most $2\chi(G)$ colors. Garey and Johnson proved this
problem is hard~\cite{GareyJ79}.  This approximation lower bound was improved
by Hastad and Zuckerman to $n^{1-\varepsilon}$ for any $\varepsilon >
0$.~\cite{Ha99,Zu07}

The next avenue is to assume the chromatic number of the input graph is known.
For example: given the promise that a graph G is 3-colorable, can one
efficiently find a coloring with 8 colors? The best would be to find a coloring
with 4 colors, but this is already known to be NP-hard.~\cite{GuKh2000} The
best known algorithms to find approximate colorings of 3-colorable graphs
regrettably depend on the size of the graph. The best algorithm to date colors
3-colorable graphs with $O(n^{0.19996})$ colors.~\cite{KawarabayashiT14} 

The lower bounds are a bit more hopeful. It is known to be NP-hard to color a
$k$-colorable graph using $2^{\sqrt[3]{k}}$ colors if $k$ is sufficiently
large~\cite{Huang13}.  There are a handful of other linear lower bounds that
work for all $k \geq 3$, but to the best of our knowledge this is the best
asymptotic result. The big open problem is to find an upper bound depending
only on $k$. Even $k^{k^k}$ colors would be considered progress.

Our topic of study in Chapter~\ref{ch:resilience} refines the approximate
coloring question in the following sense. We introduce a parameter $r \in
\mathbb{N}$, and we make the assumption that the input graph is $k$-colorable,
and \emph{remains} $k$-colorable under the adversarial addition of $r$ new
edges. We call such graphs $r$-resiliently $k$-colorable. 

The idea is that highly resilient instances are easy to color, and the learning
about the resilience boundary could provide a useful perspective on the general
approximate 3-coloring question. Moreover, since the boundary is itself an
NP-hardness boundary, completely characterizing the complexity of resilient
$k$-coloring gives a finer perspective on P vs. NP.

Following this thread, in addition to coloring, the main argument is that
resilience is a natural parameter for any combinatorial search problem. One can
formulate a resilient version of Hamiltonian path or 3D-matching or unique
games. Indeed, in Chapter~\ref{ch:resilience} we completely characterize the
complexity of resilient boolean satisfiability, where resilience means a
formula can be satisfied even under the operation of fixing variables to truth
values. In that case resilience does not introduce a difficulty gradient, and
$r$-resilient $k$-SAT is either NP-hard or vacuously empty. However, we
demonstrate that one can create resilience-preserving reductions between NP
problems, specifically reducing resilient 1-resilient 6-SAT to 1-resilient
3-coloring, which shows the latter is also NP-hard. The reduction works by
constructing a graph in such a way so that an edge added to this new graph
corresponds to a weaker constraint on the underlying SAT formula than fixing
the truth value of a variable. This suggests that there is a fruitful theory of
resilience-preserving reductions, and it extends existing tools for
constructing NP-hardness reductions to the study of resilience.

\section{MapReduce and Distributed Complexity}

In Chapter~\ref{ch:mapreduce} we turn to a model of computing based on the
paradigm of MapReduce~\cite{DeanG08}. This model distributes computation
across a sublinear number of processors and rounds in such a way that no
processor has random access to the entire input in any round. 

A central open problem related to MapReduce is whether one can determine if an
undirected graph is connected in a constant number of rounds. It can be solved
on MapReduce in $O(\log n)$ rounds~\cite{Karloff10}, but in MapReduce rounds
are the primary complexity measure, leading one to the present problem.  More
generally, many problems on sparse graphs have unresolved complexity for the
same reason.

The conjecture is that graph connectivity cannot be solved in constant rounds.
However, there are simply no tools available for proving a lower bound on the
complexity of a MapReduce problem. In the hopes of progress toward a lower
bound, one would naturally ask whether tools from classical complexity theory
could be brought to bear. As it turns out, our work is the first to give
substantial connections between MapReduce (and other modern distributed
computing models) and classical complexity classes for Turing machines. This
was due in part to informal model definitions in the literature, which we
refine, and the alternative focus on upper bounds and algorithm design in the
general community.
 
The consequences of our work have interesting practical implications and the
potential for deep connections to other parts of classical complexity. That is,
this work does not just use complexity theory to understand MapReduce, it
provides connections that go both ways, and answering questions about MapReduce
could shed light on conjectures about tradeoffs between time and space
complexity. In the rest of this section we will describe the techniques used in
Chapter~\ref{ch:mapreduce} and the basic complexity classes involved. All of
the results we describe in the remainder of this section are common knowledge,
and we refer the reader to the standard text of Arora and Barak~\cite{AroraB09}.

First, recall that $\TIME(f(n))$ is the class of decision problems solvable on
a Turing machine using $O(f(n))$ steps, and $\SPACE(g(n))$ is the class of
problems solvable on a Turing machine using $O(g(n))$ tape cells. The class
$\TISP(f(n), g(n))$ is the class of problems solvable on a Turing machine in
simultaneous time $f(n)$ \emph{and} space $g(n)$. Note that it is widely
believed that $\TIME(f(n)) \cap \SPACE(g(n)) \neq TISP(f(n), g(n))$. Very few
facts are known about $\TISP$. Further recall that $\NTIME(f(n))$ is the set of
problems solvable on a nondeterministic Turing machine in $O(f(n))$ steps, and
that $\P = \cup_i \TIME(n^i)$, $\NP = \cup_i \NTIME(n^i)$, $\EXP = \cup_i
\TIME(2^{n^i})$, $\NEXP = \cup_i \NTIME(2^{n^i})$. 

One central idea studied in complexity theory is the notion of a
\emph{hierarchy}. The presence of a hierarchy in a complexity class means that
algorithms that are allowed more resources can solve more problems. For
example, the widely known deterministic time-hierarchy theorem states that for
any function $f(n)$ and any function $g(n)$ which grows sufficiently faster
than $f(n)$ (specifically, $f(n) \log f(n) = o(g(n))$), Turing machines that
are allowed $g(n)$ time can solve problems that cannot be solved with $f(n)$
time. In other words, $\TIME(f(n)) \subsetneq \TIME(g(n))$. In general, a
hierarchy theorem is a claim about an increasing family of complexity classes
$C$ parameterized by some resource $f(n)$, which has the form ``For all
functions $g(n)$ sufficiently larger than $f(n)$, $C(f(n)) \subsetneq
C(g(n))$.'' 

While many hierarchy theorems are known, even the standard complexity classes
still have open questions related to hierarchies. For example, while there is a
known hierarchy theorem for $\TIME$, there is no known hierarchy for $\TISP(-,
g(n))$, that is, a time hierarchy theorem within a fixed space bound. Our work
gives a new connection between this question and existing complexity hypothesis
such as the Exponential Time Hypothesis~\cite{ImpagliazzoPZ01}, which states
that there exist problems (such as boolean satisfiability) that have no
subexponential-time algorithms.  Further, a hierarchy within linear-space TISP
implies a (slightly wider) hierarchy within MapReduce. We connect these
hypotheses to give a conditional hierarchy. 

\emph{Padding} is a central technique we use to prove our hierarchy theorems.
Padding allows one to use the assumption that two large complexity classes are
not equal to prove that two smaller complexity classes are not equal. As an
illustrative example, we prove that if $\EXP \neq \NEXP$, then $\P \neq
\NP$.  Suppose that $\P = \NP$ and let $L$ be a language in $\NEXP$ decided by
a nondeterministic Turing machine $M$ in time $2^{n^k}$. For each string $x \in
L$, form $L' = \{ x1^{2^{|x|^k}} : x \in L \}$, i.e., pad $x$ with an
exponential number of 1's at the end. Then an $\NP$ machine can simulate $M$ by
ignoring the padded part of the input. This takes exponential time, but that
runtime is polynomial in the size of the padded input $x1^{2^{|x|^k}}$. Since
$\P = \NP$, there is a deterministic Turing machine $M'$ that solves $L'$, and
$M'$ can be simulated by an $\EXP$ machine after adding padding, which takes
exponential time. So $\P = \NP \to \EXP = \NEXP$. We use padding to establish
our hierarchy theorems in Chapter~\ref{ch:mapreduce}.

\section{Neighbor Aid and Disaster Recovery}

In the study of infrastructure networks, a great deal of importance is placed
on the task of disaster recovery. For example, if a hurricane damages the power
grid in a region, recovery workers must decide how best to restore the network
by repairing each site. Moreover, the cost of repairing a site is cheaper if
nearby sites have already been recovered because previously recovered nodes
provide resources to the recovery effort. This phenomenon is called
\emph{neighbor aid} by~\cite{Gutfraind14}. 

More abstractly, one can fix a network $G$ and a function $f: \mathbb{N}
\to \mathbb{N}$ representing the cost of recovering a node $v$ of $G$, where the
input is the number of neighbors of $v$ that have already been recovered. Then
the goal is to determine the optimal ordering of the nodes to repair. This
problem, introduced by~\cite{Gutfraind14}, is called the Neighbor Aided Network
Installation Problem (NANIP).

Another angle is resource deployment. One can imagine a military deployment in
which a general wishes to conquer all of the cities in a network, and it is
much easier to invade a new city if many neighboring cities already house
allied tanks. Moreover, this highlights that a traversal of the network need
not be ``connected'' in any way; an optimal strategy may be to conquer two
cities on opposite sides of the network and proceed on two fronts. Indeed, we
validate this intuition in Chapter~\ref{ch:nanip} with a counterexample to a
conjecture of~\cite{Gutfraind14} and a logarithmic approximation-lower bound
for any ``connected'' traversal (we make this notion rigorous in
Section~\ref{sec:greedynanip}). 

We further refine the computational complexity of the NANIP problem, showing
that NANIP is NP-hard even if the cost function $f$ is convex decreasing, and
the hardness reduction extends to a hardness of approximation lower bound.

\section{Note}

The works presented in this dissertation were published at computer science
conferences and mathematics journals during the course of the author's graduate
studies. Specifically, the work in Chapter~\ref{ch:anticoordination} was
published as~\cite{KunPR13}, Chapter~\ref{ch:resilience} as~\cite{KunR14},
Chapter~\ref{ch:mapreduce} as~\cite{FishKLRT15}, and Chapter~\ref{ch:nanip}
as~\cite{GutfraindKLR15}.
